% ============================================
% RAPPORT TECHNIQUE - Application Mobile Wydad AC
% Cours: Développement Mobile
% École Marocaine des Sciences de l'Ingénieur (EMSI)
% ============================================

\documentclass[12pt,a4paper]{report}

% ============= PACKAGES =============
\usepackage[utf8]{inputenc}
\usepackage[T1]{fontenc}
\usepackage[french]{babel}
\usepackage{geometry}
\usepackage{graphicx}
\usepackage{fancyhdr}
\usepackage{titlesec}
\usepackage{hyperref}
\usepackage{xcolor}
\usepackage{listings}
\usepackage{booktabs}
\usepackage{array}
\usepackage{longtable}
\usepackage{enumitem}
\usepackage{tabularx}
\usepackage{float}
\usepackage{tikz}
\usepackage{tcolorbox}

% ============= CONFIGURATION =============
\geometry{margin=2.5cm}
\hypersetup{
    colorlinks=true,
    linkcolor=wacred,
    filecolor=magenta,
    urlcolor=wacred,
    pdftitle={Rapport Technique - Application Mobile Wydad AC},
    pdfauthor={ILYAS AIT MAINA}
}

% Couleurs WAC
\definecolor{wacred}{RGB}{190, 21, 34}
\definecolor{wacblack}{RGB}{26, 26, 26}
\definecolor{codegray}{rgb}{0.95,0.95,0.95}
\definecolor{codegreen}{rgb}{0,0.6,0}

% Configuration du code
\lstset{
    backgroundcolor=\color{codegray},
    basicstyle=\ttfamily\footnotesize,
    breakatwhitespace=false,
    breaklines=true,
    captionpos=b,
    commentstyle=\color{codegreen},
    keepspaces=true,
    keywordstyle=\color{wacred},
    showspaces=false,
    showstringspaces=false,
    showtabs=false,
    tabsize=2,
    frame=single,
    rulecolor=\color{wacred}
}

% ============= EN-TÊTES =============
\pagestyle{fancy}
\fancyhf{}
\fancyhead[L]{\textcolor{wacred}{\leftmark}}
\fancyhead[R]{\textcolor{wacred}{Wydad AC App}}
\fancyfoot[C]{\thepage}
\renewcommand{\headrulewidth}{0.5pt}
\renewcommand{\headrule}{\hbox to\headwidth{\color{wacred}\leaders\hrule height \headrulewidth\hfill}}

% ============= DÉBUT DU DOCUMENT =============
\begin{document}

% ============================================
% PAGE DE GARDE
% ============================================
\begin{titlepage}
    \centering
    
    % Logo EMSI en haut
    \begin{minipage}{0.25\textwidth}
        \includegraphics[width=\textwidth]{logo.png}
    \end{minipage}
    \hfill
    \begin{minipage}{0.6\textwidth}
        \centering
        {\Large \textbf{École Marocaine des Sciences de l'Ingénieur}}\\[0.2cm]
        {\large \textbf{EMSI}}\\[0.1cm]
        {\normalsize Département Informatique et Réseaux}
    \end{minipage}
    
    \vspace{1.5cm}
    
    \rule{\textwidth}{2pt}\\[0.5cm]
    {\Huge \textcolor{wacred}{\textbf{RAPPORT TECHNIQUE}}}\\[0.3cm]
    {\LARGE \textbf{Projet de Développement Mobile}}\\[0.5cm]
    \rule{\textwidth}{2pt}
    
    \vspace{1cm}
    
    {\Huge \textcolor{wacred}{\textbf{Application Mobile}}}\\[0.5cm]
    {\Huge \textbf{Wydad Athletic Club}}\\[0.3cm]
    {\Large WAC Mobile App}
    
    \vspace{1.5cm}
    
    % Cadre étudiant
    \begin{tcolorbox}[colback=red!5!white,colframe=wacred,title=\textbf{Réalisé par}]
        \centering
        {\Large \textbf{ILYAS AIT MAINA}}
    \end{tcolorbox}
    
    \vspace{0.8cm}
    
    \begin{tabular}{ll}
        \textbf{Filière :} & Ingénierie Informatique \\[0.3cm]
        \textbf{Encadrant :} & \textbf{Pr. MOSTAFA SAADI} \\[0.3cm]
        \textbf{Année Universitaire :} & 2025 - 2026 \\
    \end{tabular}
    
    \vfill
    
    {\large \textbf{Date de soumission :} Janvier 2026}
    
\end{titlepage}

% ============================================
% TABLE DES MATIÈRES
% ============================================
\tableofcontents
\newpage

% ============================================
% CHAPITRE 1: INTRODUCTION
% ============================================
\chapter{Introduction}

\section{Contexte du Projet}

Dans le cadre du cours de développement mobile dispensé à l'\textbf{École Marocaine des Sciences de l'Ingénieur (EMSI)}, j'ai été chargé de concevoir et développer une application mobile complète. J'ai choisi de créer une application dédiée au \textbf{Wydad Athletic Club (WAC)}, le club de football le plus titré du Maroc et l'un des plus grands clubs africains.

Le Wydad Athletic Club, fondé en 1937, représente plus qu'un simple club de football pour ses supporters. C'est une institution culturelle et sportive qui compte des millions de fans au Maroc et dans le monde entier. Cette application vise à offrir aux supporters wydadis une plateforme moderne et complète pour suivre leur club favori.

\section{Objectifs du Projet}

L'objectif principal de ce projet est de développer une application mobile permettant aux supporters du WAC de :

\begin{itemize}
    \item \textbf{Suivre l'actualité} du club en temps réel
    \item \textbf{Consulter l'effectif} et les statistiques des joueurs
    \item \textbf{Voir le calendrier} des matchs et les résultats
    \item \textbf{Réserver des billets} pour les matchs avec génération de QR code
    \item \textbf{Acheter des produits} officiels de la boutique WAC
    \item \textbf{Localiser les boutiques} officielles sur une carte interactive
    \item \textbf{Gérer son profil} et historique d'achats
\end{itemize}

\section{Périmètre de l'Application}

L'application couvre deux interfaces principales :

\begin{enumerate}
    \item \textbf{Interface Utilisateur (Supporter)} : Accès aux fonctionnalités de consultation, réservation et achat
    \item \textbf{Interface Administrateur} : Gestion complète du contenu (joueurs, matchs, produits, actualités, boutiques)
\end{enumerate}

% ============================================
% CHAPITRE 2: CONCEPTS REACT NATIVE
% ============================================
\chapter{Concepts React Native Utilisés}

Ce chapitre présente en détail les concepts fondamentaux de React Native qui ont été mis en œuvre dans le développement de l'application Wydad AC.

\section{Introduction à React Native}

\textbf{React Native} est un framework open-source créé par Facebook (Meta) permettant de développer des applications mobiles natives pour iOS et Android à partir d'une seule base de code JavaScript. Contrairement aux applications hybrides, React Native génère de véritables composants natifs, garantissant des performances optimales.

\subsection{Avantages de React Native}

\begin{itemize}
    \item \textbf{Cross-Platform} : Une seule codebase pour iOS et Android
    \item \textbf{Performance Native} : Les composants sont compilés en widgets natifs
    \item \textbf{Hot Reload} : Rechargement instantané pendant le développement
    \item \textbf{Large Écosystème} : Nombreuses librairies tierces disponibles
    \item \textbf{Communauté Active} : Support et documentation abondants
\end{itemize}

\section{Composants (Components)}

Les \textbf{composants} sont les blocs de construction fondamentaux de toute application React Native. Un composant est une fonction ou classe JavaScript qui retourne des éléments d'interface utilisateur.

\subsection{Composants Fonctionnels}

Dans notre application, nous utilisons exclusivement des \textbf{composants fonctionnels}, qui sont la façon moderne et recommandée de créer des composants React.

\begin{lstlisting}[language=JavaScript, caption=Exemple de composant fonctionnel]
import React from 'react';
import { View, Text, StyleSheet } from 'react-native';

const PlayerCard = ({ player }) => {
  return (
    <View style={styles.card}>
      <Text style={styles.name}>{player.name}</Text>
      <Text style={styles.position}>{player.position}</Text>
    </View>
  );
};

const styles = StyleSheet.create({
  card: {
    backgroundColor: '#fff',
    padding: 16,
    borderRadius: 8,
  },
  name: {
    fontSize: 18,
    fontWeight: 'bold',
  },
  position: {
    color: '#666',
  },
});

export default PlayerCard;
\end{lstlisting}

\subsection{Composants Réutilisables}

Notre application contient \textbf{10+ composants réutilisables} dans le dossier \texttt{src/components/} :

\begin{table}[H]
\centering
\caption{Composants réutilisables de l'application}
\begin{tabular}{|l|p{8cm}|}
\hline
\textbf{Composant} & \textbf{Description} \\
\hline
LoadingScreen & Écran de chargement avec animation spinner \\
\hline
EmptyState & Affichage quand une liste est vide \\
\hline
ErrorState & Affichage des erreurs avec bouton de réessai \\
\hline
FilterTabs & Onglets de filtrage dynamiques \\
\hline
ScreenHeader & En-tête réutilisable avec gradient WAC \\
\hline
StatusBadge & Badge coloré selon le statut \\
\hline
QuantitySelector & Sélecteur de quantité (+/-) \\
\hline
PaymentForm & Formulaire de paiement générique \\
\hline
ImagePickerButton & Bouton de sélection d'image \\
\hline
\end{tabular}
\end{table}

\section{Props (Propriétés)}

Les \textbf{props} sont le mécanisme de passage de données d'un composant parent vers un composant enfant. Elles sont en lecture seule et permettent de rendre les composants dynamiques et réutilisables.

\begin{lstlisting}[language=JavaScript, caption=Utilisation des props]
// Composant parent
<StatusBadge 
  status="paid" 
  label="Paye" 
  color="#4CAF50" 
/>

// Composant enfant (StatusBadge.js)
const StatusBadge = ({ status, label, color }) => {
  return (
    <View style={[styles.badge, { backgroundColor: color }]}>
      <Text style={styles.label}>{label}</Text>
    </View>
  );
};
\end{lstlisting}

\section{State (État)}

Le \textbf{state} représente les données dynamiques d'un composant qui peuvent changer au fil du temps. Contrairement aux props, le state est géré localement par le composant.

\subsection{useState Hook}

Le hook \texttt{useState} permet de déclarer des variables d'état dans les composants fonctionnels.

\begin{lstlisting}[language=JavaScript, caption=Exemple d'utilisation de useState]
import React, { useState } from 'react';

const TicketsScreen = () => {
  // Declaration de l'etat
  const [selectedMatch, setSelectedMatch] = useState(null);
  const [quantity, setQuantity] = useState(1);
  const [isLoading, setIsLoading] = useState(false);

  // Modification de l'etat
  const handleSelectMatch = (match) => {
    setSelectedMatch(match);
    setQuantity(1); // Reset quantity
  };

  return (
    <View>
      <Text>Match: {selectedMatch?.opponent}</Text>
      <Text>Quantite: {quantity}</Text>
    </View>
  );
};
\end{lstlisting}

\subsection{Types d'État dans l'Application}

\begin{itemize}
    \item \textbf{État de chargement} : \texttt{isLoading}, \texttt{isRefreshing}
    \item \textbf{État de données} : \texttt{players}, \texttt{matches}, \texttt{products}
    \item \textbf{État de formulaire} : \texttt{email}, \texttt{password}, \texttt{selectedSection}
    \item \textbf{État d'erreur} : \texttt{error}, \texttt{errorMessage}
    \item \textbf{État de filtrage} : \texttt{activeFilter}, \texttt{selectedCategory}
\end{itemize}

\section{Hooks}

Les \textbf{Hooks} sont des fonctions spéciales introduites dans React 16.8 qui permettent d'utiliser l'état et d'autres fonctionnalités React dans les composants fonctionnels.

\subsection{useEffect}

Le hook \texttt{useEffect} permet d'exécuter des effets de bord comme les appels API, les abonnements ou les modifications du DOM.

\begin{lstlisting}[language=JavaScript, caption=Exemple d'utilisation de useEffect]
import { useEffect, useState } from 'react';
import api from '../services/api';

const PlayersScreen = () => {
  const [players, setPlayers] = useState([]);
  const [loading, setLoading] = useState(true);

  // Effect execute au montage du composant
  useEffect(() => {
    const fetchPlayers = async () => {
      try {
        const response = await api.get('/players');
        setPlayers(response.data);
      } catch (error) {
        console.error('Erreur:', error);
      } finally {
        setLoading(false);
      }
    };

    fetchPlayers();
  }, []); // [] = execute une seule fois au montage

  return (
    <FlatList data={players} ... />
  );
};
\end{lstlisting}

\subsection{useContext}

Le hook \texttt{useContext} permet d'accéder aux valeurs d'un Context sans avoir à passer les props à travers chaque niveau de composant.

\begin{lstlisting}[language=JavaScript, caption=Utilisation de useContext]
import { useContext } from 'react';
import { AuthContext } from '../context/AuthContext';

const ProfileScreen = () => {
  // Acces au context d'authentification
  const { user, logout } = useContext(AuthContext);

  return (
    <View>
      <Text>Bienvenue, {user.name}!</Text>
      <Button title="Deconnexion" onPress={logout} />
    </View>
  );
};
\end{lstlisting}

\subsection{useCallback et useMemo}

Ces hooks sont utilisés pour l'optimisation des performances :

\begin{itemize}
    \item \texttt{useCallback} : Mémorise une fonction pour éviter les re-créations inutiles
    \item \texttt{useMemo} : Mémorise une valeur calculée
\end{itemize}

\begin{lstlisting}[language=JavaScript, caption=Optimisation avec useCallback]
const ShopScreen = () => {
  const [products, setProducts] = useState([]);
  
  // Fonction memoisee
  const handleAddToCart = useCallback((product) => {
    addToCart(product);
  }, [addToCart]);

  // Rendu optimise
  const renderItem = useCallback(({ item }) => (
    <ProductCard 
      product={item} 
      onAddToCart={handleAddToCart} 
    />
  ), [handleAddToCart]);

  return <FlatList data={products} renderItem={renderItem} />;
};
\end{lstlisting}

\section{Context API}

Le \textbf{Context API} est une fonctionnalité de React permettant de partager des données entre composants sans passer par les props à chaque niveau. C'est une alternative légère à Redux pour la gestion d'état global.

\subsection{AuthContext - Gestion de l'Authentification}

\begin{lstlisting}[language=JavaScript, caption=Implémentation du AuthContext]
import React, { createContext, useState, useEffect } from 'react';
import AsyncStorage from '@react-native-async-storage/async-storage';
import api from '../services/api';

export const AuthContext = createContext();

export const AuthProvider = ({ children }) => {
  const [user, setUser] = useState(null);
  const [token, setToken] = useState(null);
  const [isLoading, setIsLoading] = useState(true);

  // Verification du token au demarrage
  useEffect(() => {
    const checkToken = async () => {
      const storedToken = await AsyncStorage.getItem('token');
      if (storedToken) {
        setToken(storedToken);
        await fetchUser(storedToken);
      }
      setIsLoading(false);
    };
    checkToken();
  }, []);

  const login = async (email, password) => {
    const response = await api.post('/auth/login', { email, password });
    const { token, user } = response.data;
    await AsyncStorage.setItem('token', token);
    setToken(token);
    setUser(user);
  };

  const logout = async () => {
    await AsyncStorage.removeItem('token');
    setToken(null);
    setUser(null);
  };

  return (
    <AuthContext.Provider value={{ user, token, login, logout, isLoading }}>
      {children}
    </AuthContext.Provider>
  );
};
\end{lstlisting}

\subsection{CartContext - Gestion du Panier}

Le \texttt{CartContext} gère l'état du panier d'achat avec les fonctionnalités :

\begin{itemize}
    \item Ajout/suppression de produits
    \item Modification des quantités
    \item Calcul du total
    \item Persistance locale avec AsyncStorage
\end{itemize}

\section{Navigation avec React Navigation}

\textbf{React Navigation} est la solution de navigation standard pour React Native. Notre application utilise deux types de navigation combinés.

\subsection{Stack Navigator}

Le \textbf{Stack Navigator} empile les écrans les uns sur les autres, permettant une navigation avec bouton retour.

\begin{lstlisting}[language=JavaScript, caption=Configuration du Stack Navigator]
import { createNativeStackNavigator } from '@react-navigation/native-stack';

const Stack = createNativeStackNavigator();

const AuthStack = () => (
  <Stack.Navigator screenOptions={{ headerShown: false }}>
    <Stack.Screen name="Login" component={LoginScreen} />
    <Stack.Screen name="Register" component={RegisterScreen} />
  </Stack.Navigator>
);
\end{lstlisting}

\subsection{Tab Navigator}

Le \textbf{Tab Navigator} affiche une barre d'onglets en bas de l'écran pour une navigation principale.

\begin{lstlisting}[language=JavaScript, caption=Configuration du Tab Navigator]
import { createBottomTabNavigator } from '@react-navigation/bottom-tabs';

const Tab = createBottomTabNavigator();

const MainTabs = () => (
  <Tab.Navigator
    screenOptions={{
      tabBarActiveTintColor: '#BE1522',
      tabBarInactiveTintColor: '#666',
    }}
  >
    <Tab.Screen name="Accueil" component={HomeScreen} />
    <Tab.Screen name="Joueurs" component={PlayersScreen} />
    <Tab.Screen name="Matchs" component={MatchesScreen} />
    <Tab.Screen name="Billetterie" component={TicketsScreen} />
    <Tab.Screen name="Boutique" component={ShopScreen} />
  </Tab.Navigator>
);
\end{lstlisting}

\subsection{Navigation entre Écrans}

\begin{lstlisting}[language=JavaScript, caption=Navigation et passage de paramètres]
// Naviguer vers un ecran avec parametres
navigation.navigate('PlayerDetail', { playerId: 123 });

// Recuperer les parametres dans l'ecran cible
const PlayerDetailScreen = ({ route }) => {
  const { playerId } = route.params;
  // ...
};
\end{lstlisting}

\section{Gestion des Listes avec FlatList}

\textbf{FlatList} est le composant optimisé de React Native pour afficher des listes longues. Il ne rend que les éléments visibles à l'écran, améliorant considérablement les performances.

\begin{lstlisting}[language=JavaScript, caption=Utilisation de FlatList]
const ProductsList = ({ products }) => {
  const renderItem = ({ item }) => (
    <ProductCard product={item} />
  );

  return (
    <FlatList
      data={products}
      renderItem={renderItem}
      keyExtractor={(item) => item.id.toString()}
      numColumns={2}
      showsVerticalScrollIndicator={false}
      contentContainerStyle={styles.list}
      // Optimisations
      removeClippedSubviews={true}
      maxToRenderPerBatch={10}
      windowSize={5}
      // Pull to refresh
      refreshing={isRefreshing}
      onRefresh={handleRefresh}
      // Infinite scroll
      onEndReached={loadMore}
      onEndReachedThreshold={0.5}
    />
  );
};
\end{lstlisting}

\section{StyleSheet et Design}

React Native utilise un système de styles inspiré de CSS mais adapté au mobile avec \textbf{StyleSheet}.

\begin{lstlisting}[language=JavaScript, caption=Système de styles]
import { StyleSheet } from 'react-native';

const styles = StyleSheet.create({
  container: {
    flex: 1,
    backgroundColor: '#F5F5F5',
  },
  card: {
    backgroundColor: '#FFFFFF',
    borderRadius: 12,
    padding: 16,
    marginVertical: 8,
    // Ombre iOS
    shadowColor: '#000',
    shadowOffset: { width: 0, height: 2 },
    shadowOpacity: 0.1,
    shadowRadius: 4,
    // Ombre Android
    elevation: 3,
  },
  title: {
    fontSize: 18,
    fontWeight: 'bold',
    color: '#1A1A1A',
  },
  button: {
    backgroundColor: '#BE1522', // Rouge WAC
    paddingVertical: 12,
    paddingHorizontal: 24,
    borderRadius: 8,
    alignItems: 'center',
  },
});
\end{lstlisting}

\section{Appels API avec Axios}

\textbf{Axios} est utilisé pour toutes les communications HTTP avec le backend.

\begin{lstlisting}[language=JavaScript, caption=Configuration du service API]
// src/services/api.js
import axios from 'axios';
import AsyncStorage from '@react-native-async-storage/async-storage';

const api = axios.create({
  baseURL: 'http://localhost:3000',
  timeout: 10000,
  headers: {
    'Content-Type': 'application/json',
  },
});

// Intercepteur pour ajouter le token
api.interceptors.request.use(async (config) => {
  const token = await AsyncStorage.getItem('token');
  if (token) {
    config.headers.Authorization = `Bearer ${token}`;
  }
  return config;
});

// Intercepteur pour gerer les erreurs
api.interceptors.response.use(
  (response) => response,
  (error) => {
    if (error.response?.status === 401) {
      // Token expire - rediriger vers login
    }
    return Promise.reject(error);
  }
);

export default api;
\end{lstlisting}

\section{AsyncStorage - Stockage Local}

\textbf{AsyncStorage} est une API de stockage clé-valeur persistante, similaire à localStorage sur le web.

\begin{lstlisting}[language=JavaScript, caption=Utilisation d'AsyncStorage]
import AsyncStorage from '@react-native-async-storage/async-storage';

// Stocker une valeur
await AsyncStorage.setItem('token', 'eyJhbG...');

// Recuperer une valeur
const token = await AsyncStorage.getItem('token');

// Supprimer une valeur
await AsyncStorage.removeItem('token');

// Stocker un objet (serialisation JSON)
await AsyncStorage.setItem('user', JSON.stringify(userObject));
const user = JSON.parse(await AsyncStorage.getItem('user'));
\end{lstlisting}

\section{Gestion des Images}

\subsection{Affichage d'Images}

\begin{lstlisting}[language=JavaScript, caption=Composant Image]
import { Image } from 'react-native';

// Image locale
<Image source={require('../assets/logo.png')} style={styles.logo} />

// Image distante
<Image 
  source={{ uri: 'https://example.com/player.jpg' }} 
  style={styles.playerImage}
  resizeMode="cover"
/>
\end{lstlisting}

\subsection{Image Picker}

L'application utilise \texttt{expo-image-picker} pour permettre aux administrateurs d'uploader des images.

\begin{lstlisting}[language=JavaScript, caption=Sélection d'image]
import * as ImagePicker from 'expo-image-picker';

const pickImage = async () => {
  const result = await ImagePicker.launchImageLibraryAsync({
    mediaTypes: ImagePicker.MediaTypeOptions.Images,
    allowsEditing: true,
    aspect: [1, 1],
    quality: 0.8,
  });

  if (!result.canceled) {
    setImage(result.assets[0].uri);
  }
};
\end{lstlisting}

\section{Géolocalisation}

L'application utilise \texttt{expo-location} pour localiser l'utilisateur et afficher les boutiques proches.

\begin{lstlisting}[language=JavaScript, caption=Géolocalisation avec Expo]
import * as Location from 'expo-location';

const getLocation = async () => {
  // Demander la permission
  const { status } = await Location.requestForegroundPermissionsAsync();
  if (status !== 'granted') {
    Alert.alert('Permission refusee');
    return;
  }

  // Obtenir la position
  const location = await Location.getCurrentPositionAsync({});
  setUserLocation({
    latitude: location.coords.latitude,
    longitude: location.coords.longitude,
  });
};
\end{lstlisting}

\section{Animations avec Reanimated}

\textbf{React Native Reanimated} permet de créer des animations fluides et performantes.

\begin{lstlisting}[language=JavaScript, caption=Animation avec Reanimated]
import Animated, { 
  useSharedValue, 
  useAnimatedStyle, 
  withSpring 
} from 'react-native-reanimated';

const AnimatedCard = () => {
  const scale = useSharedValue(1);

  const animatedStyle = useAnimatedStyle(() => ({
    transform: [{ scale: scale.value }],
  }));

  const onPressIn = () => {
    scale.value = withSpring(0.95);
  };

  const onPressOut = () => {
    scale.value = withSpring(1);
  };

  return (
    <Animated.View style={[styles.card, animatedStyle]}>
      <Pressable onPressIn={onPressIn} onPressOut={onPressOut}>
        {/* Contenu */}
      </Pressable>
    </Animated.View>
  );
};
\end{lstlisting}

\section{Résumé des Concepts}

\begin{table}[H]
\centering
\caption{Récapitulatif des concepts React Native utilisés}
\begin{tabular}{|l|l|}
\hline
\textbf{Concept} & \textbf{Utilisation dans l'app} \\
\hline
Composants Fonctionnels & Tous les écrans et composants \\
\hline
Props & Passage de données entre composants \\
\hline
useState & État local de chaque écran \\
\hline
useEffect & Appels API, effets de bord \\
\hline
useContext & AuthContext, CartContext \\
\hline
Context API & Gestion d'état global \\
\hline
React Navigation & Navigation Stack + Tab \\
\hline
FlatList & Listes de joueurs, produits, matchs \\
\hline
StyleSheet & Design système aux couleurs WAC \\
\hline
Axios & Communication avec l'API REST \\
\hline
AsyncStorage & Stockage du token et préférences \\
\hline
Expo SDK & Location, ImagePicker, LinearGradient \\
\hline
\end{tabular}
\end{table}

% ============================================
% CHAPITRE 3: FONCTIONNALITÉS PRINCIPALES
% ============================================
\chapter{Fonctionnalités Principales}

\section{Vue d'Ensemble}

L'application Wydad AC Mobile offre un ensemble riche de fonctionnalités réparties en plusieurs modules interconnectés.

\section{Module Authentification}

\subsection{Inscription et Connexion}

\begin{itemize}
    \item Création de compte utilisateur avec validation des champs
    \item Connexion sécurisée avec email et mot de passe
    \item Authentification par token JWT
    \item Session persistante via AsyncStorage
    \item Déconnexion sécurisée
\end{itemize}

\subsection{Gestion du Profil}

\begin{itemize}
    \item Modification des informations personnelles
    \item Changement de mot de passe sécurisé
    \item Historique des commandes et réservations
\end{itemize}

\section{Module Accueil et Actualités}

\begin{itemize}
    \item \textbf{Écran d'accueil dynamique} avec le prochain match programmé
    \item \textbf{Fil d'actualités} avec plusieurs catégories (Équipe, Match, Club, Transfert)
    \item \textbf{Articles à la une} mis en avant
    \item \textbf{Détail des articles} avec images et contenu complet
\end{itemize}

\section{Module Effectif}

\begin{itemize}
    \item \textbf{Liste des joueurs} avec photos et informations de base
    \item \textbf{Filtrage par position} (Gardiens, Défenseurs, Milieux, Attaquants)
    \item \textbf{Fiche détaillée joueur} avec statistiques complètes
\end{itemize}

\section{Module Matchs}

\begin{itemize}
    \item \textbf{Calendrier des matchs} à venir
    \item \textbf{Résultats} des matchs passés
    \item \textbf{Filtrage} par compétition (Botola Pro, CAF, Coupe du Trône)
    \item \textbf{Informations détaillées} : lieu, date, heure, adversaire
\end{itemize}

\section{Module Billetterie}

\begin{itemize}
    \item \textbf{Sélection de match} parmi les matchs à venir
    \item \textbf{Choix de section du stade} avec différents tarifs
    \item \textbf{Réservation avec limite de temps} (15 minutes)
    \item \textbf{Paiement sécurisé} (simulation)
    \item \textbf{Génération QR code} unique pour chaque billet
    \item \textbf{Téléchargement PDF} du billet
\end{itemize}

\section{Module Boutique}

\begin{itemize}
    \item \textbf{Catalogue de produits} officiels WAC
    \item \textbf{Catégories} : Maillots, Vêtements, Accessoires, Écharpes
    \item \textbf{Gestion du panier} avec quantités modifiables
    \item \textbf{Processus de commande} complet
    \item \textbf{Génération de facture PDF}
\end{itemize}

\section{Module Boutiques Physiques}

\begin{itemize}
    \item \textbf{Carte interactive} avec Google Maps
    \item \textbf{Localisation GPS} de l'utilisateur
    \item \textbf{10 boutiques} dans 6 villes marocaines
    \item \textbf{Détails} : adresse, horaires, téléphone
\end{itemize}

\section{Interface Administrateur}

L'application inclut 8 écrans d'administration complets :

\begin{itemize}
    \item \textbf{Dashboard} : Statistiques globales
    \item \textbf{Gestion des joueurs} : CRUD complet avec upload d'images
    \item \textbf{Gestion des matchs} : Programmation et résultats
    \item \textbf{Gestion des produits} : Stock, prix, catégories
    \item \textbf{Gestion des actualités} : Publication et mise à jour
    \item \textbf{Gestion des boutiques} : Coordonnées GPS
    \item \textbf{Gestion des tickets} : Validation et statistiques
    \item \textbf{Gestion des réclamations} : Traitement et réponses
\end{itemize}

% ============================================
% CHAPITRE 4: ARCHITECTURE TECHNIQUE
% ============================================
\chapter{Architecture Technique}

\section{Architecture Globale}

L'application suit une architecture \textbf{Client-Serveur} classique avec séparation claire entre le frontend mobile et le backend API.

\begin{figure}[H]
\centering
\begin{tikzpicture}[
    box/.style={rectangle, draw=wacred, fill=red!10, minimum width=3cm, minimum height=1cm, text centered, rounded corners},
    arrow/.style={->, >=stealth, thick, color=wacred}
]
    \node[box] (client) at (0,0) {React Native / Expo};
    \node[above] at (0,0.7) {\textbf{Frontend Mobile}};
    
    \node[box] (api) at (5,0) {Express.js API};
    \node[above] at (5,0.7) {\textbf{Backend}};
    
    \node[box] (db) at (10,0) {SQLite};
    \node[above] at (10,0.7) {\textbf{Base de données}};
    
    \draw[arrow] (client) -- node[above] {HTTP/REST} (api);
    \draw[arrow] (api) -- node[above] {SQL} (db);
\end{tikzpicture}
\caption{Architecture globale de l'application}
\end{figure}

\section{Structure du Projet}

\subsection{Arborescence Frontend}

\begin{lstlisting}[language=bash, caption=Structure du répertoire src/]
src/
|-- components/          # 10+ composants reutilisables
|-- context/             # AuthContext, CartContext
|-- hooks/               # Hooks personnalises
|-- navigation/          # Stack + Tab Navigator
|-- screens/             # 25+ ecrans
|   |-- auth/            # Login, Register
|   |-- admin/           # 8 ecrans administration
|-- services/            # api.js (Axios)
|-- theme/               # colors.js
+-- utils/               # Utilitaires
\end{lstlisting}

\subsection{Arborescence Backend}

\begin{lstlisting}[language=bash, caption=Structure du répertoire backend/]
backend/
|-- server.js            # Point d'entree Express
|-- database.js          # Configuration SQLite
|-- middleware/          # authAdmin.js, authUser.js
|-- routes/              # 11 fichiers de routes
+-- utils/               # pdfGenerator.js
\end{lstlisting}

\section{Base de Données}

\begin{table}[H]
\centering
\caption{Tables de la base de données SQLite}
\begin{tabular}{|l|p{8cm}|}
\hline
\textbf{Table} & \textbf{Description} \\
\hline
admins & Comptes administrateurs \\
\hline
users & Utilisateurs/supporters \\
\hline
players & Effectif des joueurs WAC \\
\hline
matches & Calendrier et résultats \\
\hline
tickets & Billets réservés/achetés \\
\hline
products & Produits de la boutique \\
\hline
orders & Commandes boutique \\
\hline
news & Articles d'actualité \\
\hline
stores & Boutiques physiques \\
\hline
\end{tabular}
\end{table}

% ============================================
% CHAPITRE 5: TECHNOLOGIES UTILISÉES
% ============================================
\chapter{Technologies Utilisées}

\section{Stack Frontend}

\begin{table}[H]
\centering
\caption{Technologies Frontend}
\begin{tabular}{|l|l|p{5cm}|}
\hline
\textbf{Technologie} & \textbf{Version} & \textbf{Utilisation} \\
\hline
React Native & 0.81.5 & Framework mobile cross-platform \\
\hline
Expo & 54.0.31 & Plateforme de développement \\
\hline
React Navigation & 6.x & Navigation Stack + Tab \\
\hline
Axios & 1.6.2 & Client HTTP \\
\hline
AsyncStorage & 2.2.0 & Stockage local persistant \\
\hline
React Native Maps & 1.20.1 & Cartes Google Maps \\
\hline
React Native Reanimated & 4.1.1 & Animations \\
\hline
\end{tabular}
\end{table}

\section{Stack Backend}

\begin{table}[H]
\centering
\caption{Technologies Backend}
\begin{tabular}{|l|l|p{5cm}|}
\hline
\textbf{Technologie} & \textbf{Version} & \textbf{Utilisation} \\
\hline
Node.js & 18+ & Runtime JavaScript serveur \\
\hline
Express & 4.18 & Framework API REST \\
\hline
SQLite3 & 5.1 & Base de données \\
\hline
JWT & 9.0 & Authentification par tokens \\
\hline
bcryptjs & 2.4 & Hashage des mots de passe \\
\hline
PDFKit & 0.13 & Génération PDF \\
\hline
QRCode & 1.5 & Génération QR codes \\
\hline
\end{tabular}
\end{table}

% ============================================
% CHAPITRE 6: CAPTURES D'ÉCRAN
% ============================================
\chapter{Captures d'Écran}

\section{Écrans Utilisateur}

\textit{Note : Insérer ici les captures d'écran des principaux écrans de l'application.}

\subsection{Authentification}

\begin{figure}[H]
\centering
\fbox{\parbox{0.4\textwidth}{\centering\vspace{3cm}[Screenshot LoginScreen]\vspace{3cm}}}
\caption{Écran de connexion}
\end{figure}

\subsection{Écran d'Accueil}

\begin{figure}[H]
\centering
\fbox{\parbox{0.4\textwidth}{\centering\vspace{4cm}[Screenshot HomeScreen]\vspace{4cm}}}
\caption{Écran d'accueil avec actualités et prochain match}
\end{figure}

\subsection{Effectif des Joueurs}

\begin{figure}[H]
\centering
\fbox{\parbox{0.4\textwidth}{\centering\vspace{4cm}[Screenshot PlayersScreen]\vspace{4cm}}}
\caption{Liste des joueurs avec filtres par position}
\end{figure}

\subsection{Billetterie}

\begin{figure}[H]
\centering
\fbox{\parbox{0.4\textwidth}{\centering\vspace{4cm}[Screenshot TicketsScreen]\vspace{4cm}}}
\caption{Réservation de billets avec sélection de section}
\end{figure}

\subsection{Boutique}

\begin{figure}[H]
\centering
\fbox{\parbox{0.4\textwidth}{\centering\vspace{4cm}[Screenshot ShopScreen]\vspace{4cm}}}
\caption{Catalogue de produits officiels}
\end{figure}

% ============================================
% CHAPITRE 7: DIFFICULTÉS ET SOLUTIONS
% ============================================
\chapter{Difficultés Rencontrées et Solutions}

\section{Difficultés Techniques}

\subsection{Gestion de l'Authentification JWT}

\textbf{Problème :} Maintenir la session utilisateur persistante à travers les redémarrages de l'application.

\textbf{Solution :} Implémentation d'un AuthContext global qui stocke le token dans AsyncStorage et le vérifie au démarrage.

\subsection{Intégration de Google Maps}

\textbf{Problème :} Configuration complexe de react-native-maps avec les permissions.

\textbf{Solution :} Utilisation de expo-location pour une gestion simplifiée des permissions.

\subsection{Génération de PDF avec QR Code}

\textbf{Problème :} Générer des PDF incluant des QR codes côté serveur.

\textbf{Solution :} Combinaison de PDFKit et QRCode en Node.js.

\subsection{Performance des Listes}

\textbf{Problème :} Lenteur lors de l'affichage de longues listes.

\textbf{Solution :} Utilisation de FlatList avec optimisations (removeClippedSubviews, React.memo).

\section{Leçons Apprises}

\begin{enumerate}
    \item L'importance d'une architecture bien pensée dès le départ
    \item La valeur des composants réutilisables
    \item La nécessité de tester sur plusieurs appareils
    \item L'utilité d'un système de design cohérent
\end{enumerate}

% ============================================
% CHAPITRE 8: CONCLUSION ET PERSPECTIVES
% ============================================
\chapter{Conclusion et Perspectives}

\section{Bilan du Projet}

Ce projet m'a permis de développer une application mobile complète et fonctionnelle pour le Wydad Athletic Club. Les objectifs initiaux ont été atteints :

\begin{itemize}
    \item \textbf{Application stable} avec plus de 25 écrans fonctionnels
    \item \textbf{Backend robuste} avec 11 modules API REST
    \item \textbf{Interface utilisateur soignée} aux couleurs du WAC
    \item \textbf{Fonctionnalités avancées} : QR codes, PDF, géolocalisation
    \item \textbf{Panneau d'administration complet}
\end{itemize}

\section{Compétences Acquises}

\begin{itemize}
    \item \textbf{React Native \& Expo} : Développement mobile cross-platform
    \item \textbf{Node.js \& Express} : Création d'API REST
    \item \textbf{Gestion d'état} : Context API React
    \item \textbf{Authentification} : JWT
    \item \textbf{Base de données} : SQL
    \item \textbf{Git} : Gestion de versions
\end{itemize}

\section{Perspectives d'Amélioration}

\begin{itemize}
    \item \textbf{Notifications push} pour les matchs
    \item \textbf{Chat en direct} entre supporters
    \item \textbf{Paiement réel} avec Stripe/PayPal
    \item \textbf{Tests automatisés} avec Jest
    \item \textbf{Migration TypeScript}
\end{itemize}

\section{Mot de Fin}

Ce projet représente une expérience enrichissante qui m'a confronté aux défis réels du développement mobile. La réalisation d'une application complète m'a permis de mettre en pratique les connaissances acquises durant ma formation à l'EMSI.

Je suis reconnaissant envers mon encadrant, \textbf{Pr. MOSTAFA SAADI}, pour ses conseils et son accompagnement tout au long de ce projet.

\vspace{1cm}
\begin{center}
{\Large \textcolor{wacred}{\textbf{DIMA WYDAD}}}
\end{center}

% ============================================
% ANNEXES
% ============================================
\appendix

\chapter{Guide d'Installation}

\section{Prérequis}

\begin{itemize}
    \item Node.js 18 ou supérieur
    \item npm ou yarn
    \item Expo CLI : \texttt{npm install -g expo-cli}
    \item Application Expo Go sur smartphone
\end{itemize}

\section{Installation Backend}

\begin{lstlisting}[language=bash]
cd wydadapplication/backend
npm install
npm start
\end{lstlisting}

\section{Installation Frontend}

\begin{lstlisting}[language=bash]
cd wydadapplication
npm install
npx expo start
\end{lstlisting}

\section{Compte Admin}

\begin{lstlisting}
Email: admin@wac.ma
Mot de passe: admin123
\end{lstlisting}

% ============================================
% RÉFÉRENCES
% ============================================
\chapter*{Références}
\addcontentsline{toc}{chapter}{Références}

\begin{enumerate}
    \item Documentation React Native : \url{https://reactnative.dev/docs/getting-started}
    \item Documentation Expo : \url{https://docs.expo.dev/}
    \item Documentation Express.js : \url{https://expressjs.com/}
    \item Documentation React Navigation : \url{https://reactnavigation.org/}
    \item Site officiel du Wydad AC : \url{https://www.wydadac.ma/}
\end{enumerate}

\end{document}
